%%%%%%%%%%%%%%%%%%%%%%%%%%%%%%%%%%%%%%%%%%%%%%%%%%%%%%%%%%%%%%%%%%%%
%%  The file contains the LaTeX shell for BIRS workshop reports.
%%
%%  The reports are being compiled into one volume so you must stick
%%  to the following.
%%
%%  Do not change/add anything before \begin{document}.
%%
%%  There is a list of packages you can uncomment and use.  
%%
%%  The use of TeX/LaTeX macros is NOT permitted.  Reports with macros
%%  will not be accepted. 
%%
%%  Please let LaTeX handle the numbering of your sections, equations, 
%%  figures and tables.
%%
%%  Do not be too concerned about page breaks etc.  These may change
%%  when your document is put into the volume.  We will make sure the
%%  page breaks look okay.
%%
%%  Please use thebibliography environment to do your references.  Use
%%  the format shown in the examples below.
%%
%%  Please use the \includegraphics for including images.  If you
%%  wish, graphics can be sent separately with comments in the tex
%%  file to indicate where they should be included and how big they
%%  they should be.
%%
%%  You can find all of the reports for the BIRS workshops online
%%  under the Publications link.  Select "BIRS 2003 Proceedings" to
%%  view all the reports.  The Scientific Director recommends the
%%  following reports as 'good' examples: Chapters 2, 6, 7, 9, 14, 15
%%  and 22.
%%
%%  Last revised on 24 March 2006.
%%%%%%%%%%%%%%%%%%%%%%%%%%%%%%%%%%%%%%%%%%%%%%%%%%%%%%%%%%%%%%%%%%%%%
\documentclass[10pt]{article}

\setlength{\textwidth}{6.0in}
\setlength{\textheight}{9in}
\setlength{\oddsidemargin}{0.5in}
\setlength{\evensidemargin}{0in}
\setlength{\topmargin}{-0.25in}
\pagestyle{myheadings}

\bibliographystyle{plain}

\usepackage{times}

%%  Please uncomment any package you need to use.
%% 
%%  If you need to use any additional packages please email BIRS
%%   Programme Coordinator, birs@birs.ca, first.
%% 
%%  \usepackage{epsfig}
%% 
%%  \usepackage{amssymb}
%% 
%%  \usepackage{amsmath}
%% 
%%  \usepackage{amsthm}
%% 
%%  \usepackage{graphics}
%% 

\begin{document}

\title{The Canadian Network for Modelling Infectious Diseases: Progress and Next Steps}
\author{David Earn (McMaster University),\\ Caroline Colijn (Simon Fraser University),\\ Irena Papst (Public Health Agency of Canada)}
\date{12--17 November 2023}
\maketitle
\vspace{1in}  % This space is so the page break is approximately the
              % same as in the volume.

\section{Motivation of the Workshop}

rip from the proposal

\section{Structure of the Workshop}

rip from the schedule/planning doc

\section{Presentation Highlights}

\subsection{Special Sessions}

\being{itemize}
    \item ask Kevin and Caroline for a summary of the surveillance discussion
    \item ask Steve for a summary of the historic disease data portal talk
    \item ask Steve and Irena for a summary of the macpan2 talk and hands-on session
\end{itemize}

\subsection{Short Talks}

soliciting presentation summaries from participants that gave talks

\section{Testimonials}

also soliciting from participants

\section{Outcome of the Meeting/Conclusion}

\being{itemize}
    \item general conclusion: enthusiasm for continuing to collaborate and grow the EIDM networks in Canada
    \item mention and cite the charting the future paper
    \item will feed into the next iteration of a macpan2 workshop
\end{itemize}

\begin{thebibliography}{}

%%%  Sample references.  Please use this format.

%\bibitem{label1} C. Radin and M. Wolff, Space tilings and local isomorphism,
%\emph{Geometriae Dedicata} \textbf{42} (1992), 355--360.

%\bibitem{label2} B. Solomyak, Spectrum of dynamical systems arising
%from Delone sets. In \emph{Quasicrystals and Discrete Geometry,
%(J. Patera, ed.)}, Fields Institute Monographs, \textbf{10}, 265--275,
%American Mathematical Society, 1998.

%\bibitem{label3} P. Walters, \emph{An introduction to ergodic theory},
%Springer Graduates Texts in Mathematics, Springer-Verlag, New York, 1982.

\end{thebibliography}

\end{document}

