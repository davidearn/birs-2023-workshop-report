
Infectious disease modelling that aims to contribute to public health
decision-making has a long history, going back to Daniel Bernoulli’s
work on smallpox control in the 18th century \cite{Bern1760}.  In the
early 20th century, Ronald Ross developed a mathematical model to help
identify effective malaria control strategies \cite{Ross11}, and
Kermack and McKendrick developed the foundations of modern
mathematical epidemiology \cite{KermMcKe27}.  While theoretical work
on disease modelling continued \cite{Bart57,Bail75,AndeMay91},
attention from decision-makers and politicians was scarce until the
2001 Foot and Mouth Disease epidemic in the UK, and the SARS epidemic
in 2003.  Over the last 20 years, the perception of mathematical
modelling as a valuable tool in the public health policy process has
become more and more common, especially in the context of influenza
pandemic preparedness, which set the stage for immediate, serious
engagement with modellers when the SARS-CoV-2 pandemic exploded in
early 2020.

From the start of the pandemic, many disease modellers around the
world found themselves in constant demand from public health agencies
and policy-makers.  Recognizing the importance of this development,
NSERC invested \$10M with the aim of creating networks of disease
modellers, public health professionals, and policy-makers.  A quarter of
the investment (\$2.5M) was allocated to our
\href{https://canmod.net/}{CANMOD network} (the remaining 75\% was
allocated to four other networks).

CANMOD aims to increase Canada’s capacity for data-driven emerging
infectious disease modelling (EIDM) to directly support short, medium,
and long-term public health decisions. Our network comprises
collaborative teams of modellers, statisticians, epidemiologists,
public health decision-makers, and those implementing and delivering
interventions. The questions we are tackling are grounded in public
health needs and generated in partnership between research
investigators and knowledge users -- public health leaders, health
administrators and policy-makers. This collaborative research supports
data collection, curation and access, with the hope that it will
increase the speed with which critical information is made available.


The COVID-19 pandemic in Canada has made clear the urgent and
immediate need for modelling of local context to inform decisions that
are often implemented in local jurisdictions, and across diverse
epidemiologic and health system contexts. Effective public health
benefits from the support of engaged modellers who understand the
local data, local epidemiological, socio-cultural, and health-system
contexts, and who are passionate about collaborating on public health
research problems. Our 44 co-applicants and dozens of collaborators
have been engaged in this kind of work since the beginning of the
COVID-19 pandemic, and are ideally placed to ensure that the
collaborations we continue to build are effective. Enthusiasm from
public health institutions was clear from the immediate high level of
engagement in early 2020, the regular use of our research results in
decision-making, and from numerous and enthusiastic letters we have
received from a wide range of public health institutions across
Canada, spanning municipal, regional, provincial and national
jurisdictions. Some of the high-priority scientific questions that
have emerged through this collaborative research were discussed at the
workshop are summarized by this report.


As researchers have moved from the daily challenges of
decision-making during the COVID-19 pandemic to working on longer-term
policy questions, CANMOD continues to build and coordinate national
capacity in infectious disease modelling at the forefront of public
health. This capacity will position public health across Canada for
better control of any infectious disease, and will build better
preparedness and resilience in case of future pandemics. We are providing extensive experiential training opportunities for
postdoctoral fellows (PDFs), graduate and undergraduate students at
the intersection of infectious disease modelling, public health policy
and decision making, and we are committed to increasing equity,
diversity, and inclusion in the next generation of infectious disease
modellers. Our trainees are well-placed for quantitatively-oriented
careers in academia, industry and the public sector, both in Canada
and abroad.


CANMOD’s multi-disciplinary and multi-sectoral network is addressing
all infectious disease challenges with principles of equality,
diversity and inclusion through its recruitment, training, and
research, and through events like the November 2023 BIRS meeting that
this report summarizes.
