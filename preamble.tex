%%%%%%%%%%%%%%%%
%% TIME STAMP %%
%%%%%%%%%%%%%%%%
\usepackage{scrtime} % for \thistime (this package MUST be listed first!)
\date{\today\ @ \thistime}

%%%%%%%%%%%%%%
%% GEOMETRY %%
%%%%%%%%%%%%%%
%%\usepackage{geometry}  % See geometry.pdf to learn the layout options. There are lots.
%%\geometry{letterpaper} % ... or a4paper or a5paper or ...
%%\geometry{landscape}   % Activate for rotated page geometry

%%%%%%%%%%%
%% STYLE %%
%%%%%%%%%%%
%\usepackage[parfill]{parskip}    % Activate to begin paragraphs with an empty line rather than an indent
\usepackage{pdflscape}
% Figures to end
%% \usepackage[nomarkers,figuresonly]{endfloat}
%\usepackage[nomarkers]{endfloat}

\usepackage{placeins} % \FloatBarrier

%% line numbering
\usepackage{lineno}\renewcommand\thelinenumber{\color{gray}\arabic{linenumber}}

%%%%%%%%%%%%%%%%%%%%%%%
%% STANDARD PACKAGES %%
%%%%%%%%%%%%%%%%%%%%%%%
%% DE: I removed the dutchcal package here so we get standard mathcal.
%% To load the dutchcal font without overriding mathcal, this works:
%%https://tex.stackexchange.com/questions/507585/how-to-use-a-calligraphy-package-only-for-part-of-the-document
%% we use this to get lower case cal from dutchcal (e.g., \mathdutchcal{g}).
\DeclareMathAlphabet{\mathdutchcal}{U}{dutchcal}{m}{n} % Load the dutchcal font

\usepackage{amsmath,amsfonts,enumerate,float,fullpage,graphicx,longtable,lscape,mathtools,multirow,rotating,stmaryrd,subcaption,xcolor}

%% show labels for equations, sections, etc., in margin:
\usepackage{showlabels}

%% clever spacing or not spacing:
\usepackage{xspace}

%%%%%%%%%%%%
%% TABLES %%
%%%%%%%%%%%%

\usepackage{tabularx}
\usepackage{tabularray} % tblr (more control over table structure)
%% align numbers at decimal point in tables:
%% https://tex.stackexchange.com/questions/2746/aligning-numbers-by-decimal-points-in-table-columns
\usepackage{dcolumn}
\newcolumntype{d}[1]{D{.}{.}{#1}}

%%%%%%%%%%
%% MATH %%
%%%%%%%%%%

\newtheorem{thm}{Theorem}
\newtheorem{lem}{Lemma}
\newtheorem{prop}{Proposition}
\newtheorem{rem}{Remark}
\newtheorem{cor}{Corollary}

\DeclareMathOperator*{\Ein}{Ein}

%%%%%%%%%%%%%%
%% CLEVEREF %%
%%%%%%%%%%%%%%
%% order matters: hyperref before cleverref, and biblatex last
\usepackage{hyperref}
%% cleveref package for convenient hyper-referencing/citing:
\usepackage[nameinlink,capitalize]{cleveref}
%% The default cref format includes a space before the number, e.g.,
%% "§ 3" rather than "§3", so when using § it looks better to
%% define the cref format explicitly:
%% https://tex.stackexchange.com/questions/247538/remove-space-in-the-default-cref-command
\crefformat{section}{\S#2#1#3}
%% A space is still included when multiple sections are referenced,
%% but I think removing the space for multiple section refs would look
%% worse.
\crefname{section}{\S}{\S\S}
\Crefname{section}{Section}{Sections}
\crefname{equation}{Equation}{Equations}
\Crefname{equation}{Equation}{Equations}
\crefname{figure}{Figure}{Figures}
\Crefname{figure}{Figure}{Figures}
\Crefname{appsec}{Appendix}{appendices}

%% \citet for either named or numbered references
%% FIX: better to use \citeauthor and redefine \citet for numbered refs !!! 
%%\newcommand{\mycitet}[2]{\citet{#2}} % named
\newcommand{\mycitet}[2]{#1~\cite{#2}} % numbered

\newcommand{\needref}{{\color{red}\bfseries$\langle$NEED REF$\rangle$}}

%%%%%%%%%%%%%%
%% GRAPHICS %%
%%%%%%%%%%%%%%
\usepackage[all,color]{xy}
\DeclareGraphicsRule{.tif}{png}{.png}{`convert #1 `dirname #1`/`basename #1 .tif`.png}

\usepackage{xcolor}
\definecolor{Scol}{HTML}{4DAF4A}
\definecolor{Icol}{HTML}{E41A1C}
\definecolor{Rcol}{HTML}{377EB8}

\usepackage{framed} % framed environment (e.g., shaded)
%% shadecolour must be defined for framed environment:
\definecolor{shadecolor}{rgb}{0.9,0.9,0.9}
%%\definecolor[named]{shadecolor}{rgb}{0.9,0.9,0.9}
%%\definecolor[named]{verylightgray}{rgb}{0.95,0.95,0.95}
%%\definecolor[named]{lightgray}{rgb}{0.9,0.9,0.9}
%%\definecolor[named]{grey}{rgb}{0.7,0.7,0.7}

%%%%%%%%%%%%
%% MACROS %%
%%%%%%%%%%%%

%% Latin abbreviations
\newcommand{\ie}{\textit{i.e., }}
\newcommand{\eg}{\textit{e.g., }}
\newcommand{\etc}{\textit{etc.}}
%%\newcommand{\nb}{\textit{n.b., }}
\newcommand{\nb}{\textit{N.B., }}
\newcommand{\vs}{\textit{vs. }}
\newcommand{\cf}{\textit{cf. }}

%% abbreviations
\newcommand{\KM}{KM\xspace}

%% derivatives
%% note that \d is a built-in accent macro
\newcommand{\dee}{{\rm d}}
\newcommand{\dd}[2]{{\frac{\dee{#1}}{\dee{#2}}}}
\newcommand{\dbyd}[2]{{{\dee{#1}}/{\dee{#2}}}}
\newcommand{\ddx}[1]{\dd{#1}{x}}
\newcommand{\ddt}[1]{\dd{#1}{t}}
\newcommand{\ddtau}[1]{\dd{#1}{\tau}}
\newcommand{\dbydx}[1]{\dbyd{#1}{x}}
\newcommand{\dbydt}[1]{\dbyd{#1}{t}}
\newcommand{\dbydtau}[1]{\dbyd{#1}{\tau}}

%% initial conditions
\newcommand{\xinit}{x_{\rm i}}
\newcommand{\yinit}{y_{\rm i}}
\newcommand{\zinit}{z_{\rm i}}
%% versions with lower subscripts:
%%\newcommand{\xinit}{x_{{}_{\rm i}}}
%%\newcommand{\yinit}{y_{{}_{\rm i}}}
%%\newcommand{\zinit}{z_{{}_{\rm i}}}
%% effective initial conditions for j'th epidemic:
\newcommand{\xinitj}[1]{x_{{\rm i},#1}}
%%\newcommand{\xinitj}[1]{x_{{}_{{\rm i},#1}}}

%% special points
\newcommand{\tpeak}{t_{\rm p}}
\newcommand{\taupeak}{\tau_{\rm p}}
\newcommand{\xpeak}{x_{\rm p}}
\newcommand{\ypeak}{y_{\rm p}}
\newcommand{\zpeak}{z_{\rm p}}

%% orders of magnitude
\newcommand{\oh}{{o}} % {\mathcal o} is a vertical tilde
\newcommand{\Oh}{{\mathcal O}}

%% sets
\newcommand{\reals}{{\mathbb R}}
\newcommand{\integers}{{\mathbb Z}}
\newcommand{\naturals}{{\mathbb N}}

%% standard functions that are missing
\DeclareMathOperator{\sech}{sech}
\DeclareMathOperator{\arctanh}{arctanh}

%% heterogeneity functions
\newcommand{\Hfun}{{\mathcal H}}
\newcommand{\Kfun}{{\mathcal K}}

%% epi parameters
\newcommand{\Tau}{{\mathcal T}} % capital \tau
\newcommand{\R}{{\mathcal R}}
%%\newcommand{\Rn}{{\R_{0}}}
\newcommand{\Tinf}{T_{\rm inf}}
\newcommand{\Tlat}{T_{\rm lat}}
\newcommand{\Tgen}{T_{\rm gen}}
\newcommand{\eoR}{\epsilon}
%%\newcommand{\eoRinv}{\big(\frac{1}{\eoR}\big)}
\newcommand{\eoRinv}{\eoR^{-1}}
\newcommand{\etainv}{\eta^{-1}}

%%\newcommand{\xstar}{x^{\star}}
%%\newcommand{\ystar}{y^{\star}}
\newcommand{\xstar}{x_{\!{}_{\star}}}
\newcommand{\ystar}{y_{\!{}_{\star}}}

\newcommand{\xin}{x_{\rm in}}
\newcommand{\xH}{x_{\rm H}}

%%\newcommand{\xf}{x_{\rm f}}
%%\newcommand{\xfj}[1]{x_{{\rm f},#1}}
\newcommand{\xf}{x_{{}_{\rm f}}}
\newcommand{\xfj}[1]{x_{{}_{{\rm f},#1}}}
\newcommand{\zf}{z_{{}_{\rm f}}}

%%\newcommand{\xmax}[1]{\overline{x}_{#1}}
%%\newcommand{\xmaxj}[2]{\overline{x}_{#1,#2}}
\newcommand{\xmax}[1]{\overline{x}_{{}_{#1}}}
\newcommand{\xmaxj}[2]{\overline{x}_{{}_{#1,#2}}}

%%\newcommand{\ymax}[1]{\overline{y}_{#1}}
%%\newcommand{\ymaxj}[2]{\overline{y}_{#1,#2}}
\newcommand{\ymax}[1]{\overline{y}_{{}_{#1}}}
\newcommand{\ymaxj}[2]{\overline{y}_{{}_{#1,#2}}}

%%\newcommand{\xmin}[1]{\underline{x}_{#1}}
%%\newcommand{\xminj}[2]{\underline{x}_{#1,#2}}
%%https://tex.stackexchange.com/questions/125412/bar-below-symbol
\usepackage{stackengine}
\newcommand\barbelow[1]{\stackunder[1.2pt]{$#1$}{\rule{1ex}{.1ex}}}
\newcommand{\xmin}[1]{\barbelow{x}_{{}_{#1}}}
\newcommand{\xminj}[2]{\barbelow{x}_{{}_{#1,#2}}}

%%\newcommand{\ymin}[1]{\underline{y}_{#1}}
%%\newcommand{\yminj}[2]{\underline{y}_{#1,#2}}
\newcommand{\ymin}[1]{\barbelow{y}_{{}_{#1}}}
\newcommand{\yminj}[2]{\barbelow{y}_{{}_{#1,#2}}}

%% ART parameters
\newcommand{\Sg}{{\mathcal S}_{\rm g}} % cost to guarders from sneakers
\newcommand{\Ghat}{\widehat{G}}
\newcommand{\sneakprop}{\sigma}
\newcommand{\sneakprophat}{\hat{\sneakprop}}
\newcommand{\paternityloss}{\ell}
\newcommand{\nug}{\nu_{\rm g}}
\newcommand{\nus}{\nu_{\rm s}}
\newcommand{\Kg}{K_{\rm g}}
%%%%%%%%%%%%%%%%%%%%%%%%%%%%%%%%%%%%%%
%% ONEHALF SYMBOL WITH SIDEWAYS SLASH:
%%https://tex.stackexchange.com/questions/28866/how-to-print-frac12-by-a-single-unicode-character
%%all 3 of the following packages are needed to get \textonehalf as sideways 1/2
\usepackage[T1]{fontenc}
\usepackage{textcomp}
\usepackage{lmodern}
\newcommand{\Ghalf}{G_{{}_{\!\text{\textonehalf}}}} % Ghalf
\newcommand{\xhalf}{x_{{}_{\!\text{\textonehalf}}}} % xhalf
%% NOTE: \text causes problems with tikz-generated figures: see make_figs.R
%%%%%%%%%%%%%%%%%%%%%%%%%%%%%%%%%%%%%%
%% MnSymbol must be loaded after lmodern:
%% https://tex.stackexchange.com/questions/560963/brackets-in-equation-environment-are-being-ignored
%% FIX: what do we use this package for?
\usepackage{MnSymbol} % for \llangle; conflicts with amssymb so use amsfonts
%\newcommand{\nc}[1]{\left\llangle{#1}\right\rrangle}

\makeatletter
\newcommand{\vast}{\bBigg@{4}}
\newcommand{\Vast}{\bBigg@{5}}
\makeatother

%% between big and Big
%%https://tex.stackexchange.com/questions/67399/is-there-a-way-to-manually-set-the-height-of-a-bracket
%%\def\big#1{{\hbox{$\left#1\vbox to8.5\p@{}\right.\n@space$}}}
\makeatletter
\def\semibig#1{{\hbox{$\left#1\vbox to11\p@{}\right.\n@space$}}}
\makeatother

%% DE: revised macros using \overset:
%%\newcommand{\Yabove}{\overset{\textrm{\tiny\upshape above}}{Y}}
\newcommand{\Yabove}{Y^{\uparrow}}
%%\newcommand{\Ybelow}{\underset{\textrm{\tiny\upshape below}}{Y}}
\newcommand{\Ybelow}{Y_{\downarrow}}
\newcommand{\Yout}[2]{\overset{\textrm{\tiny\upshape out}}{Y_{#1}^{#2}}}
\newcommand{\Yin}[2]{\overset{\textrm{\tiny\upshape in}}{Y_{#1}^{#2}}}
\newcommand{\Yxb}[2]{\overset{\textrm{\tiny\upshape $x$b}}{Y_{#1}^{#2}}}
\newcommand{\Yyb}[2]{\overset{\textrm{\tiny\upshape $y$b}}{Y_{#1}^{#2}}}
%%\newcommand{\Yinscaled}[2]{\overset{\textrm{\tiny\upshape in}}{\Upsilon}_{#1}^{#2}}
\newcommand{\Yinscaled}[2]{\Upsilon_{#1}^{#2}}
\newcommand{\yscaled}{\upsilon}
\newcommand{\Ylc}[2]{\overset{\textrm{\tiny\upshape lc}}{Y}{}_{#1}^{#2}}
\newcommand{\Yrc}[2]{\overset{\textrm{\tiny\upshape rc}}{Y}{}_{#1}^{#2}}
\newcommand{\Ycor}[2]{\overset{\textrm{\tiny\upshape cor}}{Y}{}_{#1}^{#2}}
%%\newcommand{\Ynull}[1]{Y^{\O}(#1)}
%%\newcommand{\Ynull}[1]{\overset{\textrm{\tiny\upshape null}}{Y}(#1)}
%%
%%\newcommand{\Xout}[1]{X^{{\rm out},#1}}
%%\newcommand{\Xin}[1]{X^{{\rm in},#1}}
%%\newcommand{\Xlc}{X^{\rm lc}}
%%\newcommand{\Xrc}{X^{\rm rc}}
%%\newcommand{\Xcor}{X^{\rm cor}}
\newcommand{\Xout}[2]{\overset{\textrm{\tiny\upshape out}}X{}_{#1}^{#2}}
\newcommand{\Xin}[2]{\overset{\textrm{\tiny\upshape in}}X{}_{#1}^{#2}}
\newcommand{\Xxb}[2]{\overset{\textrm{\tiny\upshape $x$b}}X{}_{#1}^{#2}}
\newcommand{\Xyb}[2]{\overset{\textrm{\tiny\upshape $y$b}}X{}_{#1}^{#2}}
\newcommand{\Xlc}[2]{\overset{\textrm{\tiny\upshape lc}}{X}{}_{#1}^{#2}}
\newcommand{\Xrc}[2]{\overset{\textrm{\tiny\upshape rc}}{X}{}_{#1}^{#2}}
\newcommand{\Xcor}[2]{\overset{\textrm{\tiny\upshape cor}}{X}{}_{#1}^{#2}}
%%\newcommand{\Xleft}{X^{\rm L}}
\newcommand{\Xleft}{\overleftarrow{X}}
%%\newcommand{\Xright}{X^{\rm R}}
\newcommand{\Xright}{\overrightarrow{X}}

\newcommand{\calU}{\mathcal{U}}
\newcommand{\calY}{\mathcal{Y}}
\newcommand{\calW}{\mathcal{W}}
\usepackage{mathrsfs} % \mathscr
\newcommand{\Winv}{\mathscr{E}}

%%\newcommand{\Ccor}{C^{\rm cor}}
%%\newcommand{\Cout}{C^{\rm out}}
%%\newcommand{\Cin}{C^{\rm in}}
\newcommand{\Ccor}[2]{\overset{\textrm{\tiny\upshape cor}}{C}{}_{#1}^{#2}}
\newcommand{\Cout}[2]{\overset{\textrm{\tiny\upshape out}}{C_{#1}^{#2}}}
\newcommand{\cout}[2]{\overset{\textrm{\tiny\upshape out}}{c_{#1}^{#2}}}
\newcommand{\Cin}[2]{\overset{\textrm{\tiny\upshape in}}{C}{}_{#1}^{#2}}
\newcommand{\Cyb}[2]{\overset{\textrm{\tiny\upshape $y$b}}{C}{}_{#1}^{#2}}
\newcommand{\Cphi}{C^{\phi}}
\newcommand{\cphi}{c^{\phi}}

%%%%%%%%%%%%%%
%% COMMENTS %%
%%%%%%%%%%%%%%

%% using plain TeX to define \hide so it covers multiple paragraphs:
\long\def\hide#1{{\color{lightgray}#1}}
\newcommand{\comment}{\showcomment}
\newcommand{\showcomment}[3]{\textcolor{#1}{\textbf{[#2: }\textsl{#3}\textbf{]}}}
\newcommand{\nocomment}[3]{}
\newcommand{\todd}[1]{\comment{red}{Todd}{#1}}
\newcommand{\jd}[1]{\comment{blue}{JD}{#1}}
\newcommand{\david}[1]{\comment{cyan}{David}{#1}}
\newcommand{\dsugg}[1]{{\color{red}#1}}
\newcommand{\bmb}[1]{\comment{magenta}{BMB}{#1}}
%%\newcommand{\todo}[1]{\comment{red}{TODO}{#1}}
\newcommand{\thickredline}{{\color{red}\bigskip\begin{center}\linethickness{2mm}\line(1,0){250}\end{center}\bigskip}}
\newcommand{\thickblueline}{{\color{blue}\bigskip\begin{center}\linethickness{2mm}\line(1,0){250}\end{center}\bigskip}}
\newcommand{\term}[1]{{\bfseries\slshape #1}}
\newcommand\ttbackslash{{\tt\char`\\}}
\newcommand{\macro}[1]{{\tt\ttbackslash#1}}

