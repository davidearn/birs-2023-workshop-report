\subsubsection*{Opinion dynamics and disease: One wave or many?}

\textit{Rebecca Tyson, UBC Okanagan}

Opinion dynamics, that is, changes of opinions/behaviours in a
population arising from interactions between individuals in the
population, can have a strong effect on disease dynamics. In most
modelling efforts however, such behaviours are considered to be fixed
within a given subpopulation (divided by, e.g., age or socioeconomic
class), or altered by top-down public policies. In this talk we present
a suite of models coupling disease and opinion dynamics, and show how
the interaction between these two processes can have a profound effect
on the disease dynamics, creating, e.g., multiple epidemic waves,
changes in peak size, and changes in final size. While there is a long
history of modelling disease dynamics, the field of opinion dynamics
modelling is still fairly new. We call for more research on how best to
model opinion dynamics, particularly within the context of new diseases.
