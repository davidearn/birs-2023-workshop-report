From lkanary@yukonu.ca Tue Dec  5 09:46:13 2023
Date: Tue, 5 Dec 2023 14:46:11 +0000
From: Lisa Kanary <lkanary@yukonu.ca>
To: David Earn <earn@math.mcmaster.ca>
Subject: Re: [23w5151] BIRS workshop: e-mail list, contributions to report

----

\section*{Brief Summary of Talk}

Understanding the dynamics of disease transmission and effective
interventions is crucial due to a virus' rapid
spread. Non-Pharmaceutical Interventions (NPIs) play a vital role in
mitigating the spread of a virus through a population, especially when
pharmaceutical interventions are limited or ineffective.  Implementing
NPIs such as social distancing, face masks, hand hygiene, travel
restrictions, and quarantine measures has been essential in
controlling the pandemic and reducing the burden on healthcare
systems.

This study aims to investigate the relationship between NPIs and COVID
prevalence. By incorporating NPIs into a modeling framework, this
assessment will help determine the effectiveness of NPIs in reducing
the transmission and overall prevalence of COVID-19, and effectively,
disease in general.

To investigate the relationship between COVID prevalence and NPIs, we
employ a predator-prey Lotka-Volterra model. The predator-prey model
offers a theoretical framework that allows for the examination of the
impact of NPIs on COVID transmission dynamics and provides insights
into the effectiveness of these interventions in controlling disease
prevalence. The insights provided by these mathematical models can
inform decision-making processes for policymakers, public health
officials, and researchers, and can guide the development of targeted
interventions, helping to control the spread of the virus and mitigate
its impact on public health and society. Several methods for fitting
model coefficients will be explored in this exercise.

