\subsubsection*{Toward Support for Epidemic Preparedness via Digital Twin Data
+ ``Think Big''
Proposal}
\textit{Michael Wolfson, University of Ottawa}

An essential component for pandemic preparedness is adequate data
supporting ongoing analytical capacity. Since support for specialized
pandemic-oriented data collection (+ analytical capacity) tends to wane
between acute events, data for pandemic preparedness should, as much as
possible, be designed to be useful during quiescent periods. One such
kind of data is a realistic but synthetic ``digital twin'' that closely
resembles detailed census data but is non-identifiable, hence
non-confidential. Such digital twin data would provide individual-level
details of Canada's population by small area geography and a range of
socio-economic and infectious disease-relevant characteristics,
substantially reflecting real-world heterogeneities. In turn, these data
could provide a richly textured basis for more sophisticated infectious
disease modeling, especially with regard to contact patterns more
readily incorporated into agent-based modeling as compared to the more
usual compartment models.

Such a digital twin database could be constructed by starting first with
published census cross-tabs at the census tract level using simulated
annealing, and then synthetically matching (with replacement) individual
and household level records from census public use files, thereby
substantially preserving important kinds of correlations. In addition,
data from other key microdata files like the Canadian Community Health
Survey (CCHS), the Labour Force Survey (LFS), and the time use results
from the General Social Survey could also be synthetically matched. As
these data sources are either already in the public domain, or versions
could be so constructed, the resulting digital twin would also be
non-confidential, hence completely open data.

Constructing this digital twin database and regularly updating it would
incur considerable costs. However, it would also have a much broader
range of uses than only to support epidemic modeling. With a broad range
of users, it would be more easily sustained over time as a new and
important addition to Canada's statistical system.

In order to keep the digital twin data current, ``nowcasting'' using a
public domain version of Statistics Canada's DEMOSIM model could be
used, along with other regular monthly data sets including the LFS and
CCHS.

With the advent of an epidemic and the imposition of public health
measures such as lock downs, behaviors would change. These could be
tracked via appropriately anonymized yet still geographically detailed
real-time cell phone mobility data.

In sum, a well-conceived digital twin database could provide both a
substantially improved real-time basis for infectious disease modeling
and a substantial addition to Canada's statistical system that would
have a broad range of other uses, thereby assuring its longer term
sustainability.

Based on some of the preceding discussion in the conference, this
presentation also offered the suggestion of ``thinking big'' --
developing a brief for senior officials and governments outlining the
idea of a digital twin along with other key improvements in linkable and
coherent data flows such as infections, hospitalizations, vaccinations,
and genotyping of infections. While this kind of initiative is not
something fundable using the current Canadian and provincial granting
council structures, there is a window of opportunity with recent reports
on data from the Public Health Agency of Canada and Health Infoway, and
the 2023 federal budget proposing over half a billion dollars for health
data. There is also the recent Report of the Advisory Panel on the
Federal Research Support System which could well open up the right kind
of funding opportunities.
