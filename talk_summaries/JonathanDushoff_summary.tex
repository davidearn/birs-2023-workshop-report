
\title{Conceptual models of immunity}

Better understanding of patterns and processes underlying
cross-immunity is needed to better understand future dynamics and
burden of important diseases. Simple dynamical models of
cross-immunity can often be classified as either history-based
(classifying individuals by history of infections), or status-based
(classifying individuals by what strains they are currently
effectively immune to). There is a very strong analogy between these
classifications and those of “leaky” and “polarized” vaccine models.

The talk discussed the importance of immune-boosting – that is,
enhanced immune protection following an unsuccessful infection
challenge, and demonstrated that a model with boosting provides a
practical way to bridge between the dynamics of these two paradigms,
while arguing that this mechanism is highly biologically relevant.
Differences in such conceptual assumptions can lead both to different
estimates of vaccine effectiveness, and different predictions about
future dynamics.
