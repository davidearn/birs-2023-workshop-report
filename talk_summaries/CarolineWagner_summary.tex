\textbf{Title}: Multi-Pathogen Agent-Based Models for Disease
Surveillance and Mitigation

Caroline Wagner

\textbf{Abstract}: Understanding the dynamics of emerging infections and
the efficacy of detection technologies in the context of the endemic
circulation of other pathogens is a critical aspect of effective public
health responses against infectious diseases. The use of compartmental
models to simulate the effectiveness of different detection technologies
is complicated by the importance of heterogeneity in numerous aspects of
these systems, including underlying patterns of technology distribution
within a population and variable in-host immune responses. Compartmental
models also present challenges when modeling large numbers of
co-circulating pathogens with specific disease characteristics and
immunological rules for pathogen-pathogen interactions. In light of
this, we present Pathosim, a multi-pathogen agent-based model (ABM) that
builds on the open-source COVID-19 model Covasim. Like Covasim, Pathosim
allows for flexible population and transmission network structures, and
can simulate individual in-host viral kinetics, the implementation of
pharmaceutical interventions, and testing and quarantine procedures. In
addition, Pathosim allows for the flexible characterization of any
pathogen of interest along with the specification of immunological rules
for pathogen-pathogen interactions (i.e. cross-immunity and altered
disease course during co-infection). We then demonstrate the utility of
Pathosim in terms of simulating and modeling various detection and
surveillance systems including protocols for serosurveillance and
early-detection systems based on sequencing data.
