\documentclass[12pt]{article}

\title{Evan Mitchell - Talk Summary}
\author{}
\date{}

\begin{document}

\maketitle

How many infections, hospitalizations, and deaths did COVID-19 vaccines prevent during the pandemic in Canada? This talk presents research aimed at answering this question. A compartmental model was fit to daily infection report and hospitalization occupancy data for Ontario from the start of the pandemic through the end of the Delta wave in December 2021. We use this model to simulate counterfactual scenarios where vaccines were not present, vaccine introduction was delayed 60 days, or vaccines were 25\% less effective. Results from these simulations show that the predicted numbers of infections, hospitalizations, and deaths would be orders of magnitude larger than they actually were. To follow this up, we consider the effects of introducing a hypothetical stay-at-home order in an attempt to control these counterfactual scenarios. Our main finding from these explorations is that we would have a much easier time controlling the situation in the case of less effective vaccines than in the other two cases, suggesting that it is important to release a vaccine as early as possible during a pandemic even if that vaccine is less effective than it might be otherwise. We are currently in the process of extending these results to five other provinces: Alberta, British Columbia, Manitoba, Qu\'{e}bec, and Saskatchewan.

\end{document}