\subsubsection*{The need to evaluate existing data resources and knowledge gaps to support future needs for respiratory disease surveillance and modelling}
\textit{Michael Li, Public Health Agency of Canada}

Infectious disease surveillance and health data sharing have always
been topics of discussion, even before the SARS-CoV-2 pandemic. The
SARS-CoV-2 pandemic amplified the value of these discussions,
providing researchers and government scientists with a small glimpse
of the data possibilities—such as the high frequency of time series
data reporting positive cases, testing, sequencing, hospitalization,
and death. However, what we had for SARS-CoV-2 is still far from the
ideal data structure (e.g., linkable data, health status, etc.) needed
to learn more about the questions of interest. Before we can make
further progress, the capacity diminishes due to low-frequency/quality
reporting.

Lack of surveillance and data sharing are often viewed as the same
problem; however, it is important to recognize that they are separate
issues.  Our focus should be on learning from data, not just on the
data itself. The key is not sharing data per se, but fostering
effective collaborations to obtain information from data—referred to
as "data-info or data-knowledge sharing."

This talk proposes a pandemic and peacetime preparedness version
called the "PREP" vision, which stands for Profiling, Reflection,
Exploring alternative options, and Proof of concept. Profiling
identifies what different people want to know and the bottlenecks of
knowledge gaps.  Reflection evaluates existing resources used to seek
answers to understand what needs improvement. Exploring alternative
options goes beyond the current status quo to see what can be done to
improve access and, eventually, enhance the resources. Lastly, a proof
of concept aims to validate whether the ideas are worth implementing
using the model world.