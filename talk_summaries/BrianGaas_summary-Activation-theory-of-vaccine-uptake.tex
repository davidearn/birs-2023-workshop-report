\subsubsection*{Better modeling through chemistry: quantifying COVID vaccine hesitancy}
\textit{Brian Gaas, Government of Yukon}

Much of the literature on vaccine hesitancy focuses on whether an individual receives a vaccine. However, the rate of vaccination---the number of people getting vaccinated in a given amount of time---is equally important. This work presents a conceptual framework for understanding and predicting vaccine adoption rates, following the transition state theory of chemistry. 

The vaccine uptake framework hypothesizes people will only get vaccinated if their personal Vaccine Motivation exceeds a population-averaged Vaccine Hesitancy. Within the framework, Vaccine Motivation and Vaccine Hesitancy are functionally equivalent to temperature and activation energy, respectively, within the Arrhenius equation. The proportion of unvaccinated individuals getting vaccinated per unit time is related to the negative exponential of the Vaccine Hesitancy Ratio, defined as Vaccine Motiviation divided by Vaccine Hesitancy. Neither Vaccine Motivation nor Vaccine Hesitancy are observable, but the Vaccine Hesitancy Ratio for a given time period can be estimated as the negative log-odds of vaccination status (individuals who changed vaccination status versus individuals who did not change status within that period). Logistic regression can be used to test whether the Vaccine Hesitancy Ratio varies over time, since it has the same log-odds form. 

Dose 1 uptake rates from the Yukon (Canada) for COVID-19 were analyzed using the vaccine uptake framework. The population could be clustered into four groups of people based on how the Vaccine Hesitancy Ratio changed over time: low, medium, and high Vaccine Hesitancy Ratios, and one group who never got vaccinated. Further work could include applying clustering algorithms to better differentiate groups, identifying predictors that classify individuals into each of the four groups, and applying the vaccine uptake framework to forecast future dose uptake or uptake rates of different vaccines.