
Antigenic evolution of SARS-CoV-2 in immunocompromised hosts

Ben Ashby

Prolonged infections of immunocompromised individuals have been proposed as a
crucial source of new variants of SARS-CoV-2 during the COVID-19 pandemic.
Longitudinal sampling of SARS-CoV-2 from immunocompromised hosts reveals
evidence of accelerated adaptation relative to the wider population. In
principle, sustained within-host antigenic evolution in immunocompromised hosts
could allow novel immune escape variants to emerge more rapidly, but little is
known about how and when immunocompromised hosts play a critical role in
pathogen evolution. I will discuss how a relatively simple mathematical model
can provide powerful insights into the effects of immunocompromised hosts on the
emergence of immune escape variants. Specifically, when the pathogen does not
have to cross a fitness valley for immune escape to occur, a small number of
immunocompromised hosts have no qualitative effect on antigenic evolution at the
population-level. But if a fitness valley exists, then persistent infections of
immunocompromised individuals allow mutations to accumulate so that the fitness
valley can be traversed, thereby facilitating large jumps in antigenic space at
the population-level. Our results suggest that better genomic surveillance of
infected immunocompromised individuals and better global health equality,
including improving access to vaccines and treatments for individuals who are
immunocompromised (especially in lower- and middle-income countries), may help
to prevent the emergence of future immune escape variants of SARS-CoV-2.

