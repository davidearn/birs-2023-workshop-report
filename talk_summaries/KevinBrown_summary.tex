
{\bfseries
Association between Delayed Nursing Home Outbreak Identification and
SARS-CoV-2 Infection and Mortality in Ontario, Canada

Kevin Brown
}

Delayed outbreak identification is likely an important driver of respiratory
infection transmission in nursing homes. Most studies examining outbreak
identification have been descriptive and there are no measures of delayed
outbreak identification in nursing homes. We conducted a longitudinal cohort
study of SARS-CoV-2 outbreaks from 623 nursing homes in Ontario, Canada in the
March 1, 2020 to November 14, 2020 period prior to the rollout of COVID-19
vaccination. Our exposure was the timeliness of outbreak identification, defined
as late ($\ge3$ resident-days of infection pressure) versus early
($\le2$ resident-days of infection pressure) on the date of outbreak
identification. Residents were considered to contribute infection pressure from
2 days prior to onset to 8 days afterwards while non-residents (including staff
and visitors) were not considered to contribute infection pressure. Our outcomes
were 30-day secondary infections and mortality, defined as the proportion of at
risk residents with a laboratory-confirmed SARS-CoV-2 infection with onset
within 30-days of the outbreak identification date, and mortality among these
residents. We identified 632 SARS-CoV-2 outbreaks across 623 Ontario nursing
homes during the study period. Of these, 230 (34.3\%) outbreaks were identified
late. Outbreaks identified late had higher secondary
infections (10.3\%, 4,437/43,078) and mortality (3.2\%, 1374/43,078) compared to
outbreaks identified early (infections: 2,015/61,061, $p<0.001$, mortality: 0.9\%,
579/61,061, $p<0.001$). After adjustment for 12 nursing home risk factors, the
incidence of secondary infections in outbreaks identified late was 2.90-fold
larger than that of outbreaks identified early (OR$=$2.90, 95\%CI: 2.04, 4.13).
Each 1-person-day increase in infection pressure at the time of outbreak
identification was associated with an 1.10-fold increase in the secondary
infections (OR$=$1.10, 95\%CI: 1.08, 1.12) and a 1.07-fold increase in secondary
mortality (adjusted OR$=$1.07, 95\%CI: 1.06, 1.09). In the nursing home setting,
SARS-CoV-2 outbreaks identified late evolved to be much larger than outbreaks
identified early. The timeliness of outbreak identification can be used to
predict the trajectory of an outbreak and plan for increased staffing demands,
infection control measures, and antiviral administration, with the goal of
mitigating harms to residents.
