\subsubsection*{Academic collaborations to improve wastewater-based modelling at the Pubic Health Agency of Canada}
\textit{David Champredon, Public Health Agency of Canada}

Since the COVID-19 pandemic, the surveillance of respiratory viruses in municipal wastewater has emerged as a valuable new data source for modellers.
However, there are still many unknowns regarding the various causes that can impact the viral concentration in wastewater during the journey of viruses in the sewer system.
Understanding the processes that can affect viral concentration in wastewater is critical for epidemiological surveillance at the Public Health Agency of Canada, and it involves many scientific fields, not all represented within the Agency.
In this talk, I highlighted several projects done in collaboration with academic groups that brought their expertise to help better understand how various processes can impact viral concentration in wastewater. I also presented the different ways academic groups can collaborate with PHAC.




























