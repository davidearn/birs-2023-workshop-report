
\subsubsection*{Within-host diversity of SARS-CoV-2 across animal host
species}
\textit{Jesse Shapiro, McGill University}

Viral transmission across different host species makes eradication very challenging and also opens new evolutionary trajectories for the virus. Since the beginning of the ongoing COVID-19 pandemic, SARS-CoV-2 has been transmitted from humans to several different animal species, and novel variants of concern could plausibly evolve in a non-human animal. Previously, using available whole genome consensus sequences of SARS-CoV-2 from four commonly sampled animals (mink, deer, cat, and dog) we inferred similar numbers of transmission events from humans to each animal species but a relatively high number of transmission events from mink back to humans (Naderi et al., 2023). In a genome-wide association study (GWAS), we identified 26 single-nucleotide variants (SNVs) that tend to occur in deer, more than for any other animal, suggesting a high rate of viral adaptation to deer. Here we show that deer harbor more intra-host SNVs (iSNVs) than other animals, providing a larger pool of genetic diversity for natural selection to act upon. Deer contain more distinct viral lineages than other animals, indicating possible co-infections, but this effect is unlikely to explain the overall higher diversity within deer. Compared to other animals, iSNV frequencies in deer are skewed toward higher frequencies, which is unexpected after a recent population bottleneck or population expansion and therefore suggests that deer are sampled relatively late in the course of infection. Combined with extensive deer-to-deer transmission, the high levels of within-deer viral diversity help explain the apparent rapid adaptation of SARS-CoV-2 to deer.
