\subsubsection{\macpan Modelling Software}
\label{sec:macpan}

\textit{Steven C Walker, Irena Papst}

\macpanOrig is an \R package for compartmental modelling that was developed to provide forecasts and insights to public health agencies throughout the COVID-19 pandemic. Forecasts created with \macpanOrig were prepared for the Public Health Agency of Canada, the \href{https://covid19-sciencetable.ca/}{Ontario COVID-19 Science Table}, the World Health Organization, and Public Health Ontario. Much was learned about developing general purpose compartmental modelling software during these experiences, but the pressure to deliver regular forecasts to these organizations made it difficult to focus on the software itself. With the support of CANMOD, the \macpan project was launched to re-imagine \macpanOrig, building it from the ground up to address lessons learned while responding to a global public health emergency.

The special session on \macpan started with a presentation by Steve Walker introducing the project. Steve traced the history of \macpanOrig and \macpan, using it to argue that impactful modelling requires many interdisciplinary steps along the path from epidemiological research teams to operational decision-makers. Researchers must quickly tailor a model to an emerging public-health concern, validate and calibrate it to data, work with decision-makers to define model outputs useful for stakeholders, configure models to generate those outputs, and package up those insights in an appropriate format for stakeholders. \macpan targets bottlenecks along this path that can be solved with thoughtful software engineering. The goal is to ease the software development burden on modellers, especially when they are working on an urgent public health response, so that they can devote their time and energy to the modelling itself.

After discussing the project's history and motivation, the presentation transitioned to \newline exploring \macpan's modular model building, a key feature meant to address a commonly-encountered bottleneck in modelling. New public health concerns often demand new modules to be added to existing models. For example, as vaccines against COVID-19 were deployed, models needed to be modified to include vaccination. Even beyond responding to a public health emergency, a common paradigm in modelling (and in writing code) is to start simply and add complexity incrementally, testing outputs at every step of the way. Experience shows that it can be surprisingly difficult to add new modules to a modelling pipeline if your existing toolkit is not designed for modular model building. Steve briefly reviewed existing approaches to modular compartmental modelling based on mathematical tools from graph theory and category theory. Steve described how modules in \macpan can be represented by tables (like tables in a database), and that widely-understood table manipulation tools (like join and group-by) can be used to combine modules without the need for advanced mathematical concepts.

After the presentation, participants were invited to a hands-on session to explore \newline \macpan, led by Steve Walker, Irena Papst, and Ben Bolker. There were roughly 20 participants in the session, representing a wide range of career phases, from graduate students to tenured faculty. We started the session by helping participants download and install the software on their computers. There were several installation hiccups that we were able to troubleshoot on the fly. These issues gave us valuable insight into potential difficulties deploying this tool more widely, and have inspired further work on the software.

We then invited participants to work through a \href{https://github.com/canmod/macpan2/blob/4b9812d4b05f3e959f2f84f4712d72a681b06e6c/vignettes/quickstart.Rmd}{getting started vignette} to enable them to further familiarise themselves with the software's model specification grammar, which enables modular model building. The vignette walks users through specifying a very simple epidemiological model, and then introduces software features that make it easy to add additional structure to models, such as modules for multiple infection types (\eg asymptomatic, symptomatic), multiple locations (often referred to as ``metapopulation'' models), stratification by vaccination status, and more. The vignette specifically works through the example of specifying a two-strain model while demonstrating \macpan functions key to easily specifying ``structured'' models.

After participants worked through the vignette, some worked on specifying other models in \macpan, as a way to test their understanding of the model specification grammar and to experiment with other features of the package. Two participants worked together to try to specify a Lotka-Volterra predator-prey model, and their attempts revealed interesting points of friction in the software that have directly inspired further development. These attempts also spurred the addition of Lotka-Volterra models to \macpan's model library. Three participants independently provided the same feedback about how calibrating models to data is a bigger bottleneck than modular model building, which has inspired us to make existing calibration tools more accessible via the \macpan interface.

This session was the first \macpan training ever run, and overall, we received a lot of valuable feedback from participants on it. We continue to use this feedback to both improve guides for the software, as well as the software itself.